\section{Followup Tools and Services (Melissa)} \label{sec:followup}
% Materials used
%LSST CBW Participants Drive > Presentations - Friday Morning
%SOAR_AEON_Presentation_June2019.pdf
%LCO LSST Broker LONI Presentation

\MLG[inline]{Section status: MLG preliminary text generation in progress. Should invite Cesar and Rachel and Austin to contribute here.}

\smallskip

\noindent
{\it Ensure this section contains:
\begin{itemize}
    \item How will automated follow-up be enabled?
    \item How will brokers and science platforms integrate and coordinate with
follow-up observing facilities and networks?
\end{itemize}
}

Time-domain science with LSST will benefit significantly from additional observations enabled by \emph{follow-up} facilities and services, and some science goals absolutely require follow-up data. Enabling rapid follow-up observations is one of the main science-driven needs for community brokers, as described in Section \ref{sec:science}. In these sessions we explored how to enable automated, real-time integration of the LSST alert stream with follow-up networks and target observation managers (TOMs).

\subsection{Follow-up Networks}

Astronomical Event Observing Network\footnote{\url{https://lco.global/aeon/}} (AEON) is the software architecture that allows for the automatic queue scheduling of requested follow-up observations, and is currently in development between the Las Cumbres Observatory and SOAR Telescope. With AEON, accepted proposals may submit their observation requests via API, thereby allowing the auto-generation of follow-up observations by, e.g., TOMs, for targets which meet certain pre-defined criteria. For example, three consecutive epochs of multi-band photometry show a faint but rapidly rising blue transient associated with a low-$z$ host galaxy. A request for a low-resolution spectrum could be automatically generated with, e.g., a 2-day window and intermediate priority, and submitted to the SOAR scheduler. In this process, AEON is the software that enables the TOM to communicate directly with the SOAR queue. 

\MLG[inline]{from here we could get more detailed on how AEON works -- Cesar's talk has many details -- but I'll leave that for later.}

Some of the main challenges faced by such follow-up networks are minimizing overheads (e.g., idle time, network delays, observations that run over their scheduled time window), ensuring the necessary instrument and telescope constraints can be specified in the TOM, support for moving objects, and the automatic reduction and resubmission of processed follow-up data from the telescopes back to the TOMs. 

\subsection{TOMs}

\smallskip
\noindent
{\it Ensure this section contains:
\begin{itemize}
    \item Will TOMs be run as a service that brokers can connect to?
    \item Will brokers integrate TOMS into their platform astropy style?
\end{itemize}
}

In the LSST era, sample sizes for transients and variable stars will increase by at least an order of magnitude for even the rarest sub-groups, and likely much more, as will the diversity in classification types -- all of which will have different science-driven needs for follow-up observations and analysis. Target Observation Managers (TOMs) are currently being developed to meet this challenge. TOMs could be considered as a specialized sort of science platform that is driven by time-domain science and follow-up observations: software to coordinate observing programs and keep track of target samples, observations, and derived data products. 

The Las Cumbres Observatory's TOM Toolkit aims to provide an open-source modular software package that enables astronomers to build TOM systems and customize them for a given science goals. These software packages provide core functions for common tasks and well-defined interfaces to allow the TOM to communicate with, e.g., brokers and follow-up networks. By making this open-source, the hope is that a community of TOM users will arise and support each other in the further development and use of TOM systems. 

In the example above of a faint but rapidly rising blue transient, the TOMs role might be ingesting the associated alerts from the broker, performing additional tasks to add information to the target (e.g., cross-matches with external catalogs, light-curve feature fits, adding tags or labels, assigning to science-based samples), evaluating a target's need for science-driven follow-up, auto-generating follow-up observation requests and submitting them to follow-up networks, and ingesting the follow-up data (e.g., spectra) and performing analysis (e.g., spectral typing) which then further informs the need for science-driven follow-up.

\MLG[inline]{Some of the main challenges faced by TOMs are ... (MLG is unsure, the LCO TOM Toolkit talk didn't have a list like the AEON talk did... but surely they could contribute here).}

To  address these challenges, we make the following recommendations: 