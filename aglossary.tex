% DO NOT EDIT - generated by /Users/leanne/LSST/repos/lsst-texmf/texmf/../bin/generateAcronyms.py from https://lsst-texmf.lsst.io/.
\newacronym{AEON} {AEON} {Astronomical Event Observatory Network}
\newglossaryentry{AGN} {name={AGN}, description={}}
\newglossaryentry{AMPEL} {name={AMPEL}, description={}}
\newglossaryentry{ANTARES} {name={ANTARES}, description={Arizona-NOAO Temporal Analysis and Response to Events System}}
\newglossaryentry{APC} {name={APC}, description={}}
\newacronym{API} {API} {Application Programming Interface}
\newglossaryentry{AURA} {name={AURA}, description={\gls{Association of Universities for Research in Astronomy}}}
\newacronym{AWS} {AWS} {Amazon Web Services}
\newglossaryentry{AZ} {name={AZ}, description={}}
\newglossaryentry{Alert} {name={Alert}, description={A packet of information for each source detected with signal-to-noise ratio > 5 in a difference image during Prompt Processing, containing measurement and characterization parameters based on the past 12 months of LSST observations plus small cutouts of the single-visit, template, and difference images, distributed via the internet}}
\newglossaryentry{Alert Production} {name={Alert Production}, description={The principal component of Prompt Processing that processes and calibrates incoming images, performs Difference Image Analysis to identify DIASources and DIAObjects, packages and distributes the resulting Alerts, and runs the Moving Object Processing System}}
\newglossaryentry{Archive} {name={Archive}, description={The repository for documents required by the NSF to be kept. These include documents related to design and development, construction, integration, test, and operations of the LSST observatory system. The archive is maintained using the enterprise content management system DocuShare, which is accessible through a link on the project website www.project.lsst.org}}
\newglossaryentry{Archive Center} {name={Archive Center}, description={Part of the LSST Data Management System, the LSST archive center is a data center at NCSA that hosts the LSST Archive, which includes released science data and metadata, observatory and engineering data, and supporting software such as the LSST Software Stack}}
\newglossaryentry{Association Pipeline} {name={Association Pipeline}, description={An application that matches detected Sources or DIASources or generated Objects to an existing catalog of Objects, producing a (possibly many-to-many) set of associations and a list of unassociated inputs. Association Pipelines are used in Prompt Processing after DIASource generation and in the final stages of Data Release processing to ensure continuity of Object identifiers}}
\newglossaryentry{Association of Universities for Research in Astronomy} {name={Association of Universities for Research in Astronomy}, description={ consortium of US institutions and international affiliates that operates world-class astronomical observatories, AURA is the legal entity responsible for managing what it calls independent operating Centers, including LSST, under respective cooperative agreements with the National Science Foundation. AURA assumes fiducial responsibility for the funds provided through those cooperative agreements. AURA also is the legal owner of the AURA Observatory properties in Chile}}
\newglossaryentry{B} {name={B}, description={Byte (8 bit)}}
\newglossaryentry{BACODINE} {name={BACODINE}, description={}}
\newglossaryentry{BATSE} {name={BATSE}, description={}}
\newglossaryentry{Broker} {name={Broker}, description={Software which receives and redistributes Alerts, and may also perform processing such as filtering for certain characteristics, cross-matching with non-LSST catalogs, and/or light-curve classification, in order to identify and prioritize targets for follow-up and/or make scientific analyses. }}
\newglossaryentry{CA} {name={CA}, description={}}
\newacronym{CADC} {CADC} {Canadian Astronomy Data Centre}
\newglossaryentry{CBW} {name={CBW}, description={}}
\newglossaryentry{Center} {name={Center}, description={An entity managed by AURA that is responsible for execution of a federally funded project}}
\newglossaryentry{Commissioning} {name={Commissioning}, description={A two-year phase at the end of the Construction project during which a technical team a) integrates the various technical components of the three subsystems; b) shows their compliance with ICDs and system-level requirements as detailed in the LSST Observatory System Specifications document (OSS, LSE-30); and c) performs science verification to show compliance with the survey performance specifications as detailed in the LSST Science Requirements Document (SRD, LPM-17)}}
\newglossaryentry{Construction} {name={Construction}, description={The period during which LSST observatory facilities, components, hardware, and software are built, tested, integrated, and commissioned. Construction follows design and development and precedes operations. The LSST construction phase is funded through the \gls{NSF} \gls{MREFC} account}}
\newglossaryentry{Contract} {name={Contract}, description={A binding legal agreement between parties obligating the one (typically the  'seller') to furnish certain supplies or services and the other (typically, the buyer) to compensate the seller for the supplies or services with some form of consideration, (typically money). The term, 'contract' is used interchangeably with 'sub-award' 'agreement' 'memorandum of understanding and/or agreement' and 'purchase order' Each is a term used to differentiate between a purchase-order-format type document and a complex purchase in a subcontract/sub-award-format type document. These also include awards and notices of awards; job orders or task letters issued under basic ordering agreements; letter contracts; orders, such as purchase orders and subcontracts under which the order becomes effective by written acceptance or performance; and bilateral contract modifications}}
\newacronym{DCR} {DCR} {Differential Chromatic Refraction}
\newglossaryentry{DE} {name={DE}, description={}}
\newglossaryentry{DEC} {name={DEC}, description={Declination}}
\newacronym{DESC} {DESC} {Dark Energy Science Collaboration}
\newacronym{DIA} {DIA} {Difference Image Analysis}
\newglossaryentry{DIAObject} {name={DIAObject}, description={A DIAObject is the association of DIASources, by coordinate, that have been detected with signal-to-noise ratio greater than 5 in at least one difference image. It is distinguished from a regular Object in that its brightness varies in time, and from a SSObject in that it is stationary (non-moving)}}
\newglossaryentry{DIASource} {name={DIASource}, description={A DIASource is a detection with signal-to-noise ratio greater than 5 in a difference image}}
\newacronym{DM} {DM} {\gls{Data Management}}
\newacronym{DMS} {DMS} {Data Management Subsystem}
\newglossaryentry{DMTN} {name={DMTN}, description={DM Technical Note}}
\newacronym{DOE} {DOE} {\gls{Department of Energy}}
\newacronym{DR} {DR} {Data Release}
\newacronym{DRP} {DRP} {Data Release Production}
\newglossaryentry{Data Management} {name={Data Management}, description={The LSST Subsystem responsible for the Data Management System (DMS), which will capture, store, catalog, and serve the LSST dataset to the scientific community and public. The DM team is responsible for the DMS architecture, applications, middleware, infrastructure, algorithms, and Observatory Network Design. DM is a distributed team working at LSST and partner institutions, with the DM Subsystem Manager located at LSST headquarters in Tucson}}
\newglossaryentry{Data Management Subsystem} {name={Data Management Subsystem}, description={The Data Management Subsystem is one of the four subsystems which constitute the LSST Construction Project. The Data Management Subsystem is responsible for developing and delivering the LSST Data Management System to the LSST Operations Project.}}
\newglossaryentry{Data Management System} {name={Data Management System}, description={The computing infrastructure, middleware, and applications that process, store, and enable information extraction from the LSST dataset; the DMS will process peta-scale data volume, convert raw images into a faithful representation of the universe, and archive the results in a useful form. The infrastructure layer consists of the computing, storage, networking hardware, and system software. The middleware layer handles distributed processing, data access, user interface, and system operations services. The applications layer includes the data pipelines and the science data archives' products and services}}
\newglossaryentry{Data Release} {name={Data Release}, description={The approximately annual reprocessing of all LSST data, and the installation of the resulting data products in the LSST Data Access Centers, which marks the start of the two-year proprietary period}}
\newglossaryentry{Data Release Production} {name={Data Release Production}, description={An episode of (re)processing all of the accumulated LSST images, during which all output DR data products are generated. These episodes are planned to occur annually during the LSST survey, and the processing will be executed at the Archive Center. This includes Difference Imaging Analysis, generating deep Coadd Images, Source detection and association, creating Object and Solar System Object catalogs, and related metadata}}
\newglossaryentry{Department of Energy} {name={Department of Energy}, description={cabinet department of the United States federal government; the DOE has assumed technical and financial responsibility for providing the LSST camera. The DOE's responsibilities are executed by a collaboration led by SLAC National Accelerator Laboratory}}
\newglossaryentry{Difference Image} {name={Difference Image}, description={Refers to the result formed from the pixel-by-pixel difference of two images of the sky, after warping to the same pixel grid, scaling to the same photometric response, matching to the same PSF shape, and applying a correction for Differential Chromatic Refraction. The pixels in a difference thus formed should be zero (apart from noise) except for sources that are new, or have changed in brightness or position. In the LSST context, the difference is generally taken between a visit image and template. }}
\newglossaryentry{Difference Image Analysis} {name={Difference Image Analysis}, description={The detection and characterization of sources in the Difference Image that are above a configurable threshold, done as part of Alert Generation Pipeline}}
\newglossaryentry{Differential Chromatic Refraction} {name={Differential Chromatic Refraction}, description={The refraction of incident light by Earth's atmosphere causes the apparent position of objects to be shifted, and the size of this shift depends on both the wavelength of the source and its airmass at the time of observation. DCR corrections are done as a part of DIA}}
\newglossaryentry{DocuShare} {name={DocuShare}, description={The trade name for the enterprise management software used by LSST to archive and manage documents}}
\newglossaryentry{Document} {name={Document}, description={Any object (in any application supported by DocuShare or design archives such as PDMWorks or GIT) that supports project management or records milestones and deliverables of the LSST Project}}
\newglossaryentry{END} {name={END}, description={}}
\newglossaryentry{EPO} {name={EPO}, description={Education and Public Outreach}}
\newglossaryentry{Education and Public Outreach} {name={Education and Public Outreach}, description={The LSST subsystem responsible for the cyberinfrastructure, user interfaces, and outreach programs necessary to connect educators, planetaria, citizen scientists, amateur astronomers, and the general public to the transformative LSST dataset}}
\newacronym{FAQ} {FAQ} {Frequently Asked Question}
\newacronym{FITS} {FITS} {\gls{Flexible Image Transport System}}
\newglossaryentry{Flexible Image Transport System} {name={Flexible Image Transport System}, description={an international standard in astronomy for storing images, tables, and metadata in disk files. See the IAU FITS Standard for details}}
\newacronym{GCN} {GCN} {GRB Coordinates Network}
\newglossaryentry{GIT} {name={GIT}, description={}}
\newglossaryentry{GRB} {name={GRB}, description={Gamma-Ray Burst}}
\newglossaryentry{HOW} {name={HOW}, description={}}
\newacronym{HTC} {HTC} {High Throughput Computing}
\newglossaryentry{Handle} {name={Handle}, description={The unique identifier assigned to a document uploaded to DocuShare}}
\newacronym{IAU} {IAU} {International Astronomical Union}
\newglossaryentry{ID} {name={ID}, description={}}
\newglossaryentry{IV} {name={IV}, description={}}
\newglossaryentry{IVOA} {name={IVOA}, description={International Virtual-Observatory Alliance}}
\newglossaryentry{LAFS} {name={LAFS}, description={}}
\newacronym{LCO} {LCO} {Las Cumbres Observatories}
\newglossaryentry{LDM} {name={LDM}, description={LSST Data Management (Document Handle)}}
\newglossaryentry{LIGO} {name={LIGO}, description={}}
\newglossaryentry{LOI} {name={LOI}, description={}}
\newglossaryentry{LONI} {name={LONI}, description={}}
\newglossaryentry{LPG} {name={LPG}, description={}}
\newglossaryentry{LPM} {name={LPM}, description={LSST Project Management (Document Handle)}}
\newglossaryentry{LSE} {name={LSE}, description={LSST Systems Engineering (Document Handle)}}
\newglossaryentry{LSSTC} {name={LSSTC}, description={\gls{LSST} Corporation. an Arizona 501(c)3 not-for-profit corporation formed in 2003 for the purpose of designing, constructing, and operating the LSST System. During design and development, the Corporation stewarded private funding used for such essential contributions as early site preparation, mirror construction, and early data management system development. During construction, LSSTC will secure private operations funding from international affiliates and play a key role in preparing the scientific community to use the LSST dataset}}
\newglossaryentry{MARS} {name={MARS}, description={}}
\newglossaryentry{MLG} {name={MLG}, description={}}
\newacronym{MOPS} {MOPS} {Moving Object Processing System}
\newglossaryentry{MOS} {name={MOS}, description={}}
\newglossaryentry{MPC} {name={MPC}, description={}}
\newglossaryentry{MREFC} {name={MREFC}, description={\gls{Major Research Equipment and Facility Construction}}}
\newglossaryentry{Major Research Equipment and Facility Construction} {name={Major Research Equipment and Facility Construction}, description={the NSF account through which large facilities construction projects such as LSST are funded}}
\newglossaryentry{Moving Object Processing System} {name={Moving Object Processing System}, description={The Moving Object Processing System (MOPS) identifies new SSObjects using unassociated DIASources. MOPS is part of the Science Pipelines}}
\newglossaryentry{NASA} {name={NASA}, description={National Aeronautics and Space Administration}}
\newglossaryentry{NCSA} {name={NCSA}, description={National Center for Supercomputing Applications}}
\newglossaryentry{NEO} {name={NEO}, description={Near-Earth Object}}
\newglossaryentry{NOAO} {name={NOAO}, description={National Optical Astronomy Observatories (USA)}}
\newacronym{NSF} {NSF} {\gls{National Science Foundation}}
\newglossaryentry{NY} {name={NY}, description={}}
\newglossaryentry{National Science Foundation} {name={National Science Foundation}, description={primary federal agency supporting research in all fields of fundamental science and engineering; NSF selects and funds projects through competitive, merit-based review}}
\newglossaryentry{OSS} {name={OSS}, description={Observatory System Specifications; LSE-30}}
\newglossaryentry{Object} {name={Object}, description={In LSST nomenclature this refers to an astronomical object, such as a star, galaxy, or other physical entity. E.g., comets, asteroids are also Objects but typically called a Moving Object or a Solar System Object (SSObject). One of the DRP data products is a table of Objects detected by LSST which can be static, or change brightness or position with time}}
\newglossaryentry{Operations} {name={Operations}, description={The 10-year period following construction and commissioning during which the LSST Observatory conducts its survey}}
\newglossaryentry{Opportunity} {name={Opportunity}, description={The degree of exposure to an event that might happen to the benefit of a program, project, or other activity. It is described by a combination of the probability that the opportunity event will occur and the consequence of the extent of gain from the occurrence, or impact. There are two levels of opportunities. At the macro level, a project itself is the manifestation of the pursuit of an opportunity. At the element level, tactical opportunities exist, whereby certain events, if realized, provide a cost or schedule savings to the project or increase technical performance}}
\newglossaryentry{PPDB} {name={PPDB}, description={}}
\newacronym{PSF} {PSF} {Point Spread Function}
\newglossaryentry{Project Manager} {name={Project Manager}, description={The person responsible for exercising leadership and oversight over the entire LSST project; he or she controls schedule, budget, and all contingency funds}}
\newglossaryentry{Prompt Processing} {name={Prompt Processing}, description={The processing that occurs at the Archive Center on the nightly stream of raw images coming from the telescope, including Difference Imaging Analysis, Alert Production, and the Moving Object Processing System. This processing generates Prompt Data Products}}
\newacronym{QA} {QA} {Quality Assurance}
\newacronym{QC} {QC} {Quality Control}
\newglossaryentry{QDS} {name={QDS}, description={}}
\newglossaryentry{Qserv} {name={Qserv}, description={Proprietary Database built by SLAC for LSST}}
\newglossaryentry{Quality Assurance} {name={Quality Assurance}, description={All activities, deliverables, services, documents, procedures or artifacts which are designed to ensure the quality of DM deliverables. This may include \gls{QC} systems, in so far as they are covered in the charge described in LDM-622. Note that contrasts with the LDM-522 definition of “QA” as “Quality Analysis”, a manual process which occurs only during commissioning and operations. See also: Quality Control}}
\newglossaryentry{Quality Control} {name={Quality Control}, description={Services and processes which are aimed at measuring and monitoring a system to verify and characterize its performance (as in LDM-522). Quality Control systems run autonomously, only notifying people when an anomaly has been detected. See also Quality Assurance}}
\newacronym{RA} {RA} {Right Ascension}
\newglossaryentry{REC} {name={REC}, description={}}
\newglossaryentry{Review} {name={Review}, description={Programmatic and/or technical audits of a given component of the project, where a preferably independent committee advises further project decisions, based on the current status and their evaluation of it. The reviews assess technical performance and maturity, as well as the compliance of the design and end product with the stated requirements and interfaces}}
\newacronym{SC} {SC} {Science Collaboration}
\newacronym{SED} {SED} {\gls{Spectral Energy Distribution}}
\newglossaryentry{SEDM} {name={SEDM}, description={}}
\newglossaryentry{SLAC} {name={SLAC}, description={No longer an acronym; formerly Stanford Linear Accelerator Center}}
\newglossaryentry{SMWLV} {name={SMWLV}, description={}}
\newglossaryentry{SN} {name={SN}, description={}}
\newglossaryentry{SOAR} {name={SOAR}, description={}}
\newglossaryentry{SRD} {name={SRD}, description={LSST Science Requirements; LPM-17}}
\newglossaryentry{SSSC} {name={SSSC}, description={}}
\newglossaryentry{Science Collaboration} {name={Science Collaboration}, description={An autonomous body of scientists interested in a particular area of science enabled by the LSST dataset, which through precursor studies, simulations, and algorithm development lays the groundwork for the large-scale science projects the LSST will enable.  In addition to preparing their members to take full advantage of LSST early in its operations phase, the science collaborations have helped to define the system's science requirements, refine and promote the science case, and quality check design and development work}}
\newglossaryentry{Science Pipelines} {name={Science Pipelines}, description={The library of software components and the algorithms and processing pipelines assembled from them that are being developed by DM to generate science-ready data products from LSST images. The Pipelines may be executed at scale as part of LSST Prompt or Data Release processing, or pieces of them may be used in a standalone mode or executed through the LSST Science Platform. The Science Pipelines are one component of the LSST Software Stack}}
\newglossaryentry{Science Platform} {name={Science Platform}, description={A set of integrated web applications and services deployed at the LSST Data Access Centers (DACs) through which the scientific community will access, visualize, and perform next-to-the-data analysis of the LSST data products}}
\newglossaryentry{Software Stack} {name={Software Stack}, description={Often referred to as the LSST Stack, or just The Stack, it is the collection of software written by the LSST Data Management Team to process, generate, and serve LSST images, transient alerts, and catalogs. The Stack includes the LSST Science Pipelines, as well as packages upon which the DM software depends. It is open source and publicly available}}
\newglossaryentry{Solar System Object} {name={Solar System Object}, description={A solar system object is an astrophysical object that is identified as part of the Solar System: planets and their satellites, asteroids, comets, etc. This class of object had historically been referred to within the LSST Project as Moving Objects}}
\newglossaryentry{Source} {name={Source}, description={A single detection of an astrophysical object in an image, the characteristics for which are stored in the Source Catalog of the DRP database. The association of Sources that are non-moving lead to Objects; the association of moving Sources leads to Solar System Objects. (Note that in non-LSST usage "source" is often used for what LSST calls an Object.)}}
\newglossaryentry{Spectral Energy Distribution} {name={Spectral Energy Distribution}, description={the radiated energy of an astrophysical object as a function of energy (or wavelength) across the entire spectrum of light}}
\newglossaryentry{Subsystem} {name={Subsystem}, description={A set of elements comprising a system within the larger LSST system that is responsible for a key technical deliverable of the project}}
\newglossaryentry{Subsystem Manager} {name={Subsystem Manager}, description={responsible manager for an LSST subsystem; he or she exercises authority, within prescribed limits and under scrutiny of the Project Manager, over the relevant subsystem's cost, schedule, and work plans}}
\newglossaryentry{Systems Engineering} {name={Systems Engineering}, description={an interdisciplinary field of engineering that focuses on how to design and manage complex engineering systems over their life cycles. Issues such as requirements engineering, reliability, logistics, coordination of different teams, testing and evaluation, maintainability and many other disciplines necessary for successful system development, design, implementation, and ultimate decommission become more difficult when dealing with large or complex projects. Systems engineering deals with work-processes, optimization methods, and risk management tools in such projects. It overlaps technical and human-centered disciplines such as industrial engineering, control engineering, software engineering, organizational studies, and project management. Systems engineering ensures that all likely aspects of a project or system are considered, and integrated into a whole}}
\newacronym{TAP} {TAP} {Table Access Protocol}
\newglossaryentry{TOM} {name={TOM}, description={}}
\newglossaryentry{TOMS} {name={TOMS}, description={}}
\newglossaryentry{TVS} {name={TVS}, description={}}
\newglossaryentry{TVSSC} {name={TVSSC}, description={}}
\newacronym{US} {US} {United States}
\newglossaryentry{USA} {name={USA}, description={}}
\newglossaryentry{VIRGO} {name={VIRGO}, description={}}
\newacronym{VO} {VO} {Virtual Observatory}
\newglossaryentry{VTP} {name={VTP}, description={}}
\newglossaryentry{WA} {name={WA}, description={}}
\newglossaryentry{WFIRST} {name={WFIRST}, description={}}
\newacronym{ZTF} {ZTF} {Zwicky Transient Facility}
\newglossaryentry{airmass} {name={airmass}, description={The pathlength of light from an astrophysical source through the Earth's atmosphere. It is given approximately by sec z, where z is the angular distance from the zenith (the point directly overhead, where airmass = 1.0) to the source}}
\newglossaryentry{algorithm} {name={algorithm}, description={A computational implementation of a calculation or some method of processing}}
\newglossaryentry{astronomical object} {name={astronomical object}, description={A star, galaxy, asteroid, or other physical object of astronomical interest. Beware: in non-LSST usage, these are often known as sources}}
\newglossaryentry{calibration} {name={calibration}, description={The process of translating signals produced by a measuring instrument such as a telescope and camera into physical units such as flux, which are used for scientific analysis. Calibration removes most of the contributions to the signal from environmental and instrumental factors, such that only the astronomical component remains}}
\newglossaryentry{camera} {name={camera}, description={An imaging device mounted at a telescope focal plane, composed of optics, a shutter, a set of filters, and one or more sensors arranged in a focal plane array}}
\newglossaryentry{declination} {name={declination}, description={Often abbreviated Dec, it is a part of an equatorial coordinate pair that expresses the angular distance (usually expressed in degrees) from the Celestial Equator, measured along great circles that intersect the Equatorial poles. Positions south of the equator are given negative sign}}
\newglossaryentry{flux} {name={flux}, description={Shorthand for radiative flux, it is a measure of the transport of radiant energy per unit area per unit time. In astronomy this is usually expressed in cgs units: erg/cm2/s}}
\newglossaryentry{metadata} {name={metadata}, description={General term for data about data, e.g., attributes of astronomical objects (e.g. images, sources, astroObjects, etc.) that are characteristics of the objects themselves, and facilitate the organization, preservation, and query of data sets. (E.g., a FITS header contains metadata)}}
\newglossaryentry{monitoring} {name={monitoring}, description={In DM QA, this refers to the process of collecting, storing, aggregating and visualizing metrics}}
\newglossaryentry{pipeline} {name={pipeline}, description={A configured sequence of software tasks (Stages) to process data and generate data products. Example: Association Pipeline}}
\newglossaryentry{postage stamp} {name={postage stamp}, description={Image cutouts that are ~30x30 arcseconds, centered on an Object, and included in every Alert}}
\newglossaryentry{provenance} {name={provenance}, description={Information about how LSST images, Sources, and Objects were created (e.g., versions of pipelines, algorithmic components, or templates) and how to recreate them}}
\newglossaryentry{shape} {name={shape}, description={In reference to a Source or Object, the shape is a functional characterization of its spatial intensity distribution, and the integral of the shape is the flux. Shape characterizations are a data product in the DIASource, DIAObject, Source, and Object catalogs}}
\newglossaryentry{stack} {name={stack}, description={a grouping, usually in layers (hence stack), of software packages and services to achieve a common goal. Often providing a higher level set of end user oriented services and tools}}
\newglossaryentry{transient} {name={transient}, description={A transient source is one that has been detected on a difference image, but has not been associated with either an astronomical object or a solar system body}}
