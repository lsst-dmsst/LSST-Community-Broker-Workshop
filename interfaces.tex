% Identified needs and connect them to science 
% To enab;e X science - need Y science 
\section{User Facing Interfaces (Melissa, Gregory)}  \label{sec:interfaces}
% Resources used:
% https://community.lsst.org/t/lsst-answers-to-community-broker-faqs/3780
% LSST CBW Participants Drive > Presentations - Wednesday Morning > Bellm_AP_overview_190619.pdf


\smallskip
\noindent
\MLG{\it Could the title be just "User Interfaces"?}

\smallskip
\noindent

\MLG[inline]{\it Section status:\\ so far I've just worked in relevant content from the FAQ document; incorporating workshop materials like talk slides and notes comes next.
}

\smallskip
\noindent
{\it Summarize science-driven user interfaces once Section 2 is done. Summarize the general types of proposed interfaces from the LOI once Section 5 is done.}

\smallskip
\noindent
{\it Make sure the contents address the following:
\begin{itemize}
    \item How will the scientific community interact with community brokers and access the LSST data products? 
    \item How will community brokers interface to science platforms, data archives 
    \item How can ipelines for User-Generated data processing and analysis interface with the LSST alert stream
\end{itemize}
}

\smallskip

Users will not be able to subscribe to full, unfiltered alert streams coming directly from LSST. Instead, it is expected that most scientists will access LSST alerts via community brokers. Each community broker defines its own user interfaces through which individuals can access the broker's alert stream filters, value-added data products, queryable database, etc. This access might be via, e.g., web browsers or desktop clients.

{\bf (Discuss here some of the current brokers' user interfaces.)}


\subsection{IVOA}\label{ssec:interfaces_ivoa}
{\it What interfaces do we have with the IVOA? VO Event format?}

\smallskip

More information about VOEvents, the IVOA (International Virtual Observatory Alliance), and VTP (VOEvent Transport Protocol) can be found in \citet{2011ivoa.spec.0711S} and \citet{2017arXiv170901264A}.


\subsection{The LSST Science Platform}\label{ssec:interfaces_lsp}
{\it What will the interface between the LSP and Community Brokers look like for users? How about for the LSST Alert Filtering Service?}

\smallskip

The LSST Science Platform (LSP) is a collaborative research environment that provides access to LSST data products, and services for all science users and project staff. The LSP provides a set of integrated web applications and computational resources deployed at LSST Data Access Centers (DACs). The LSP is composed of three main aspects: (1) Portal, for exploratory search, analysis, and visualization of the LSST archive; (2) Notebook, for in-depth next-to-the-data analysis with pre-installed and custom libraries that enable the creation of added-value user-generated data products, without the need for users to download large volumes of data; and (3) Web API, enabling remote access to the LSST datasets and services using community-accepted formats and protocols\footnote{A ``VO First" strategy has been adopted, which means that IVOA protocols will be used wherever possible}. More information about the LSP design can be found in \citet{LSE-319,LDM-542}.

Here we describe alerts-related user interfaces with the LSP, from the perspective of individual users (\S~\ref{sssec:interfaces_lsp_individual}) and community brokers (\S~\ref{sssec:interfaces_lsp_brokers}).

\subsubsection{For Individual Users}\label{sssec:interfaces_lsp_individual}

As previously mentioned, it is expected that most scientists will access LSST alerts via community brokers. Although it will be possible for users to ingest a set of broker-filtered alerts to their user account in the LSST Science Platform (LSP) -- and note that this would be subject to the same data storage limits as any other uploaded data set -- we want to emphasize that a more efficient use of the LSP would be to directly access the original Prompt data products from which the alert packets are derived (i.e., the images and catalogs described in Section 3 of \citealt{LSE-163}). The contents of the Alerts and the Prompt products database are essentially identical, with the main difference being timescale: Alerts are released within $60$ seconds of image readout, whereas the Prompt products are available within $24$ hours.

The LSST Alert Filtering Service will provide basic, limited capacity access to the LSST alert stream; it is sized to allow 100 simultaneous user-generated filters to return 20 alerts per visit (Section 2.2.4, \citealt{LSE-61}). The LSST Alert Filtering Service is expected to provide VOEvent format alerts (or similar; Section 3.5.2 of \citealt{LSE-163}). {\bf (Re. last two sentences, could instead cite section of this paper that describes LAFS.)} It is expected that users of the LSST Science Platform (LSP) will be able to define an alert stream filter in the LSP environment and have it installed in the LSST Alert Filtering Service, which is separate from the LSP. (LSP facilities are for analysis and queries of the LSST data products and not for continuously-running processes such as alert stream filters.)  Users may receive their filtered alerts from the LSST Alert Filtering Service by, e.g., a simple User Interface provided in the LSP via the Portal Aspect (Section 3.9 of \citealt{LDM-542}; Section 2.9.5 of \citealt{LDM-554}), and/or a direct connection using standard IVOA protocols to a third-party system (e.g., VTP to a private server). It is important to note that although it should be possible to query the Alerts Database from the LSP \citep{LDM-542}, the Alerts Database might only support queries by alert ID. 

\subsubsection{For Community Brokers}\label{sssec:interfaces_lsp_brokers}

External entities (e.g., brokers) may interface with the PPDB via the Web API aspect of the LSST Science Platform, and use the TAP interface\footnote{More information about TAP can be found at:
\url{http://www.ivoa.net/documents/TAP/} and \url{https://www.cadc-ccda.hia-iha.nrc-cnrc.gc.ca/en/doc/tap/}.} to query the PPDB catalogs by, e.g., using {\tt DIAObject} or {\tt DIASource} IDs from the alert packet as keys (\citealt{LSE-319,LDM-542,LDM-554}). The PPDB catalogs are updated with new data within {\tt L1PublicT} of image acquisition (currently 24h; \citealt{LSE-29}). 



\subsection{External Science Platforms and Archives}\label{ssec:interfaces_other}
{\it Integration with other science archives or platforms, e.g Datalab, Canadian LSST Advanced Science Platform, others. Interfaces to data archives such as the CADC, etc.}

% Materials used
% LSST CBW Participants Drive > Presentations - Thursday Afternoon > Stefan van der Walt - SkyPortal
% LSST CBW Participants Drive > Presentations - Thursday Afternoon > DataLab_CBW.pptx


As astronomical data sets grow in size and complexity, it becomes increasingly necessary to co-locate the data with the tools and compute resources for scientific analysis. Thus, science platforms will play critical roles in the analysis of new and archival data sets, enabling science and expanding discovery space for a broad cross-section of the community \citep[e.g.,][]{2019arXiv190305130O}. The LSST Science Platform is just one example of an astronomical science platform serving as a user interface to a data archive or an alert stream, and some community brokers and TOMs could be considered science platforms themselves. It is expected that other external science platforms may want to offer users the opportunity to access the LSST alerts and/or other LSST data products (not necessarily by having every platform ingest the LSST data, but e.g., by web APIs). In general, it is common for Science Platforms's user interfaces to be a browser-based web front-end built on a web API (such as we have mentioned here), but alternative clients might be developed {\bf (are there examples of non-browser web clients for science platforms?)}. 

One of the key concepts discussed at the CBW is for user interfaces to be modular and customizable to the user -- or a user groups' -- particular science needs. The TOM Toolkit ({\bf cite the section of this paper where TOM Toolkit is discussed}) is a great example of customizable user interfaces. SkyPortal\footnote{\url{https://skyportal.io}} \citep{skyportal2019} is also designed as a platform for time-domain survey data from ZTF and eventually LSST, allows users to customize how the data for a given object is rendered. For example, interactive plots of light-curves and postage stamp images from an alert packet can appear alongside user-supplied spectra and/or comments.

The NOAO Data Lab \citep{2019arXiv190800664O} has built its user interfaces on the guiding principle of enabling efficient exploration through which users can develop intuition through interaction with catalogs, images, and spectra, and have this lead to discovery. Embedded data visualization tools play a big role in this, and they are offered to users via, e.g., pre-installed libraries that can be called from Jupyter Notebooks. The Data Lab currently hosts public datasets from massive ground-based surveys in the NOAO Archive, and also provides a platform for ANTARES filter development and testing, and for the analysis of ZTF alerts that pass filters installed in ANTARES. {\bf (Add more detail about HOW this works, exactly?)}

The Canadian Astronomy Data Center\footnote{\url{http://www.cadc-ccda.hia-iha.nrc-cnrc.gc.ca/en/}} (CADC) offers specialized astronomy and data management expertise to the worldwide community, to support data access to and analysis of very large astronomical data sets (many, but not all, of which are from Canadian facilities). The CADC's web browser interface allows users to search its archival collection by spatial, temporal, spectral, observational constrains (e.g., proposal keywords, instrumental filter), or calibration level (e.g., raw, calibrated, or products like stacks). Authorized users can also create virtual machines (VMs) and submit batch processing jobs and share the results {\bf (unsure whether browser based?)}. The CADC is exploring ways to partner with a broker, ingest alerts (potentially subsets of alerts, and/or with greater latency), serve them to the community, and enable user activities that add value to the alerts via co-analysis with the CADC's other archival resources. 

%%% MLG: This is probably TMI for this section.
%Common themes regarding science platform development -- which are not necessarily limited to the development of their user interfaces -- surfaced at the CBW meeting, such as:
%\begin{itemize}
%\item {\bf open development} -- to encourage collaboration in the community and ensure that the tools developed are widely applicable
%\item {\bf open source} -- to reduce redundancy in tool development and enable customization
%\item {\bf open access} -- to ensure tools and data are accessible to as broad a community as possible; i.e., user accounts that are authenticated, but freely available
%\item {\bf customizable} -- to enable broad adoption of tools, they should be adaptable to different science use-cases
%\item {\bf provenance} -- mainly referring to packages and tools that can be restored to earlier versions for testing, but also tracking user contributions to tools and derived data products for attributing credit properly
%\item {\bf documentation} -- as always, tools need to be well tested and documented for broad adoption
%\end{itemize}



To  address these challenges, we make the following recommendations: 
