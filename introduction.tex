% Leanne 
\section{Introduction (Leanne)} \label{sec:intro}

{\it Section Status: draft in progress. 
To Do: 
Introduction to transient science with LSST and the challenges posed.  
Brief outline of the LSST strategy - point to section \ref{sec:alertproduction}
Introduce the concept of Community Event Brokers.   
Describe the goals and format of the meeting including the unconference sessions. }

Following the January 2019 'Call for Letters of Intent for Community Alert Brokers'  
\citep{LDM-612}
the LSST Project hosted a workshop with the goal of bringing together representatives of all 

The workshop addressed several topics: 
\begin{itemize}
    \item Science Drivers (section \ref{sec:science}) --- what are the science drivers for Brokers?
    \item Broker Architecture and Technology (section \ref{sec:archandtech}) --- what technology will underpin Broker services 
    \item Algorithms \ref{sec:algorithms}) ---  what kinds of user algorithms do Brokers need to support 
    \item Follow-up infrastructure \ref{sec:followup}) --- how do Brokers interface to follow-up services 
    \item User-facing services \ref{sec:interfaces}) --- how will the scientific community interact with brokers 
\end{itemize}

This paper summarizes the presentations and discussions of the workshop. It highlights many ideas and recommendations that came out of the workshop. The workshop presentations are available at $<someurl>$ and we refer to them throughout this paper.


To address these challenges, we make the following recommendations
% A registry of brokers should be used to facilitate communications such as those
% outlined in ticket:
% DM-20561 Updates for communications with broker proposers (from CBW FAQ)

\nrec{brokerregistry}
{Agency, Technologist}
{Long}
{cbra}
{Investigate how to facilitate setting up a registry of Brokers}
{In order to enable science users to interact with and learn more about brokers and their progress, it was proposed in an Unconference session that some kind of public registry of brokers be created.}