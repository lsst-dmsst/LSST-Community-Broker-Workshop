
% Leanne to Ensure that the community broker workshop paper includes a statement about the number of brokers we can support and the technical implications 

\begin{abstract}
Beginning late 2022, the Vera C. Rubin Observatory Legacy Survey of Space and Time ( LSST), will produce a nightly stream of ten million public alerts that will disseminate new information about  transient  , variable, and moving objects within 60 seconds.  A number of \emph{Community Brokers} will be selected to receive the full  LSST  alert stream, adding additional scientific value, and providing science users the ability to identify targets of interest and trigger follow-up observations.  In June 2019, the  LSST  project hosted a workshop on \emph{Community Brokers} in Seattle,  WA,  USA. The overarching goal of the workshop was to bring together Rubin Observatory Project personnel, representatives of the  LSST  Science Collaborations, proposers of community brokers and follow-up facilities to develop a vision of an integrated global ecosystem for  transient   science with the  LSST  alert stream. Topics of discussion included, scientific use cases and expectations for the  LSST  Prompt data products,  LSST -broker interfaces, architecture and technology choices, the broker selection process, policy issues, and development progress and lessons learnt from precursor surveys. This paper provides a summary of that workshop and a road-map towards a building a global integrated ecosystem for time-domain science in the  LSST era.

%This document was co-authored during the workshop and edited in months following. It captures the discussions and highlights many ideas that came out of the workshop. 
\end{abstract}



